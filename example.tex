%! Author = Ivan Chizhov
%! Date = 08.11.2022

% Preamble
\documentclass[12pt, minted]{civarticle}
\usepackage{blindtext}

\title{
    Пример использования класса `civarticle`\\
    \small{(оформление окружений цветом и пр.)}
}
\author{И.~В.~Чижов}
\setabstract{%
    \Blindtext[2][1]
}
\keywords{%
    \Blindtext[1][1]
}
% Document
\begin{document}
    \section{Теорема Пифагора}
    \label{sec:thm-pif}
    \begin{definition}
        \label{def:triangle}
        \emph{Прямоугольным треугольником} называется треугольник, у которого один угол равен 90 градусам.
    \end{definition}
    \begin{theorem}
        Сумма квадратов катетов равна квадрату гипотенузы, т.е.
        \begin{equation}
            \label{eq:thm-pif}
            a^2+b^2=c^2.
        \end{equation}
    \end{theorem}
    \begin{proof}
        Доказательство очень простое.
    \end{proof}
    Множество решений уравнения~\eqref{eq:thm-pif} называется пифагоровыми тройками.
    Обозначим их символом $\mathcal{C}$ (каллиграфическое $C$).
    Очевидно, что $\mathcal{C}\subseteq \mathbb{R}^{3}$.
    А это множество $\mathcal{B}$ (каллиграфическое $B$).

    \blindmathtrue
    \Blindtext[11][1]


    \section{Ссылки на разные научные статьи}
    \label{sec:ref-to-articles}

    Статья~\cite{ahlswede1977} посвящена геометрии булевого куба.

    Статья~\cite{ahmed2013} рассказывает о конструкции генератора псевдослучайных чисел на основе теории кодов, исправляющих ошибки.


    \section{Гиперссылки}
    \label{sec:hyper}

    Ссылка на раздел~\ref{sec:thm-pif} с теоремой Пифагора.
    А формула~\eqref{eq:thm-pif} описывает суть теоремы Пифагора.

    Чтобы увидеть ссылки на разные научные статьи, смотри раздел \hyperref[sec:ref-to-articles]{\textquote{Ссылки на разные научные статьи}}.

    А вот ссылка на \href{https://ru.wikipedia.org}{википедию}.


    \section{Пакет minted}
    \label{sec:minted}
%    \show\MINTED
    \if \MINTED\empty
%       do nothing, minted is off
    \else
        \inputminted{python}{code.py}
    \fi
\end{document}