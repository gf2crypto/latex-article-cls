%! Author = Ivan Chizhov
%! Date = 08.11.2022

% Preamble
%%%% ОПЦИИ
%%%% colorthm - цветные окружения: теоремы, утверждений и прочее
\documentclass[colorthm]{../civarticle}
\usepackage[math]{blindtext}

\title{
    Пример использования класса `civarticle`\\
    \large{(оформление окружений цветом и пр.)}
}
\author{И.~В.~Чижов}

% Document
\begin{document}
\blindmathtrue

%%% Титульная страница
\maketitle

\begin{abstract}
  \Blindtext[2]
\end{abstract}

\keywords{%
  \Blindtext[1]
}

% \thispagestyle{empty}
% \newpage
\tableofcontents
\par
\vspace{3\bigskipamount}
\par
% \newpage
%%%%%%%%%%%%%%%%%%%%%%%%

\Blindtext[1]
\section{Теорема Пифагора}
\label{sec:thm-pif}

\begin{definition}\label{def:example.tex:triange}
  \emph{Прямоугольным треугольником} называется треугольник, у
  которого один угол равен \(90^{\circ}\).
\end{definition}

\Blindtext[2]

\begin{definition*}
  \emph{Прямоугольным треугольником} называется треугольник, у
  которого один угол равен 90 градусам.
\end{definition*}

\blindmathtrue\blindmathpaper

\begin{remark} \label{rem:example.tex:numbered}
  Это замечание с номером.
  \blindtext%
\end{remark}

\Blindtext[2]

\begin{remark*}
  Замечание без номера.

  \Blindtext[2]
\end{remark*}

\Blindtext[2]

\begin{example}\label{ex:example:myexample}
  Это пример с номером.
  Рассмотрим замечание~\ref{rem:example.tex:numbered}.
  \Blindtext[2]
\end{example}

\Blindtext[2]

\blinditemize%

\blindenumerate%

\blinddescription%

\begin{example*}
  Пример без номера.
  \Blindtext[2]
\end{example*}

\Blindtext[2]

\begin{theorem}
  Сумма квадратов катетов равна квадрату гипотенузы, т.е.
  \begin{equation}
    \label{eq:thm-pif}
    a^2+b^2=c^2.
  \end{equation}
\end{theorem}
\begin{proof}
  Доказательство очень простое.
\end{proof}

\begin{theorem}\label{thm:example:noproof}
  В этой теореме нет доказательства.
  \Blindtext[2]
\end{theorem}

\Blindtext[2]

Множество решений уравнения~\eqref{eq:thm-pif} называется пифагоровыми
тройками.  Обозначим их символом $\mathcal{C}$ (каллиграфическое $C$).
Очевидно, что $\mathcal{C}\subseteq \mathbb{R}^{3}$.  А это множество
$\mathcal{B}$ (каллиграфическое $B$).

\textit{Курсивный шрифт}.  \textbf{Жирный шрифт}.
\textit{\textbf{Курсивный жирный шрифт}}.


\begin{proposition}\label{prop:example:numbered}
  Утверждение с номером.
  \Blindtext[2]
\end{proposition}

\begin{proposition*}
  Утверждение без номера.
  \Blindtext[2]
\end{proposition*}

\Blindtext[2]

\begin{lemma}\label{lem:example:numbered}
  Это лемма с номером.
  \Blindtext[2]
\end{lemma}

\begin{lemma*}
  Это лемма без номера.
  \Blindtext[2]
\end{lemma*}

\begin{theorem}\label{thm:example:withlemm}
  Это теорема, в доказательстве которой содержится лемма.
  \Blindtext[2]
\end{theorem}
\begin{proof}
  Доказательство разобъем на несколько лемм.

  \begin{lemma}\label{lem:example:in-1}
    Это первая внутренняя лемма.
    \Blindtext[2]
  \end{lemma}

  \Blindtext[2]

  \begin{lemma}\label{lem:example:in-2}
    Это вторая лемма.
    \Blindtext[2]
  \end{lemma}

  \begin{lemma}\label{lem:example:in-3}
    Это третья лемма.
    \Blindtext[2]
  \end{lemma}

  \Blindtext[2]
 \end{proof}

 \Blindtext[11][1]

 \begin{corollary}\label{cor:example:num}
   Это следствие с номером.
   \Blindtext[2]
 \end{corollary}

 \Blindtext[2]

 \begin{corollary*}
   Это следствие без номера.
   \Blindtext[2]
 \end{corollary*}

 \Blindtext[2]


\section{Ссылки на разные научные статьи}
\label{sec:ref-to-articles}

\begin{definition}
  \label{def:001}
  \Blindtext[2]
\end{definition}

\begin{definition*}
  \Blindtext[2]
\end{definition*}

\begin{remark}
  \label{rem:001}
  \Blindtext[2]
\end{remark}

\begin{remark*}
  \Blindtext[2]
\end{remark*}

\begin{example}
  \label{ex:001}
  \Blindtext[2]
\end{example}

\begin{example*}
  \Blindtext[2]
\end{example*}

\begin{task}
  \label{task:001}
  \Blindtext[2]
\end{task}

\begin{task*}
  \Blindtext[2]
\end{task*}

\begin{exercise}
  \label{exer:001}
  \Blindtext[2]
\end{exercise}

\begin{exercise*}
  \Blindtext[2]
\end{exercise*}

\begin{problem}
  \label{prob:001}
  \Blindtext[2]
\end{problem}

\begin{problem*}
  \Blindtext[2]
\end{problem*}

\begin{question}
  \label{ques:001}
  \Blindtext[2]
\end{question}

\begin{question*}
  \Blindtext[2]
\end{question*}

\begin{theorem}
  \label{thm:001}
  \Blindtext[2]
\end{theorem}

\begin{theorem*}
  \Blindtext[2]
\end{theorem*}

\begin{proposition}
  \label{prop:001}
  \Blindtext[2]
\end{proposition}

\begin{proposition*}
  \Blindtext[2]
\end{proposition*}

\begin{assertion}
  \label{ass:001}
  \Blindtext[2]
\end{assertion}

\begin{assertion*}
  \Blindtext[2]
\end{assertion*}

\begin{corollary}
  \label{cor:001}
  \Blindtext[2]
\end{corollary}

\begin{corollary*}
  \Blindtext[2]
\end{corollary*}

\begin{lemma}
  \label{lem:001}
  \Blindtext[2]
\end{lemma}

\begin{lemma*}
  \Blindtext[2]
\end{lemma*}

А это ссылки:

\begin{itemize}
\item Определение~\ref{def:001}
\item Замечание~\ref{rem:001}
\item Пример~\ref{ex:001}
\item Задача~\ref{task:001}
\item Упражнение~\ref{exer:001}
\item Проблема~\ref{prob:001}
\item Вопрос~\ref{ques:001}
\item Теорема~\ref{thm:001}
\item Утверждение~\ref{prop:001}
\item Утверждение~\ref{ass:001}
\item Леммы~\ref{lem:001}
\item Следствие~\ref{cor:001}
\end{itemize}

Статья~\cite{ahlswede1977} посвящена геометрии булевого куба.

Статья~\cite{ahmed2013} рассказывает о конструкции генератора
псевдослучайных чисел на основе теории кодов, исправляющих ошибки.


\section{Гиперссылки}
\label{sec:hyper}

Ссылка на раздел~\ref{sec:thm-pif} с теоремой Пифагора.  А
формула~\eqref{eq:thm-pif} описывает суть теоремы Пифагора.

Чтобы увидеть ссылки на разные научные статьи, смотри раздел
\hyperref[sec:ref-to-articles]{\textquote{Ссылки на разные научные
    статьи}}.

А вот ссылка на \href{https://ru.wikipedia.org}{википедию}.


\section{Пакет minted}
\label{sec:minted}
% \show\MINTED
\if \MINTED\empty
% do nothing, minted is off
\else \inputminted{python}{code.py} \fi
\end{document}
%%% Local Variables:
%%% mode: latex
%%% TeX-master: t
%%% End:
